\documentclass{article}
\usepackage{graphicx}
\usepackage{wrapfig}
\usepackage{xcolor}
\usepackage{amsmath}
\usepackage{verbatim}
\usepackage{makeidx}
\usepackage{float}
\usepackage{subfig}
\usepackage[left=1in,top=1in,right=1in]{geometry}

\title{Simulating the Mobot and Linkbot}  
\author{Kevin Gucwa\\Mechanical and Aerospace Engineering}
\date{\today} 
\makeindex

\begin{document}

\begin{center}
{\Huge\sf\bf RoboSim User's Guide}\\
\vspace*{2.5cm}
{\Large\bf Version 1.1.0}
\vspace{4in}

Copyright \copyright\ \today\ by UC Davis C-STEM Center, All rights reserved.

\end{center}

%\maketitle
\newpage
\tableofcontents
\newpage

\section{Introduction}
\texttt{RoboSim} is a robot simulation environment, developed by the UC Davis
Center for Integrated Computing and STEM Education (C-STEM) {\color{blue} \bf
(http://c-stem.ucdavis.edu)}, for programming Barobo Mobot and Linkbot.  The
same Ch program  can control hardware robots or virtual robots in RoboSim
without any modification.

\section{RoboSim GUI}
\label{sec:gui}
RoboSim can be conveniently launched by double clicking its icon
\includegraphics[height=24pt]{images/robosim} on the desktop.  The RoboSim
graphical user interface (GUI), shown in Figure \ref{fig:gui}, allows the user
to change between hardware and virtual robots when a Ch robot program is
executed.  There is no save button within the GUI, all changes made are
automatically saved.
\begin{figure}[H]
	\begin{center}
		\includegraphics[width=6in]{images/gui}
	\end{center}
	\caption{The RoboSim GUI.}
	\label{fig:gui}
\end{figure}

\subsection{Platform}
The {\bf Platform} entry as shown in Figure \ref{fig:platform}, allows the user
to decide whether a Ch program controls the hardware or virtual robots.  Each
time a new Ch program is started, it will check the setup based on this entry.
For a Ch robot program to control a virtual robot, check the box for {\bf
Simulated Robots}.  If the box for {\bf Hardware Robots} is checked, a Ch
program will control the physical hardware robots.
\begin{figure}[H]
	\begin{center}
		\includegraphics[width=3in]{images/platform}
	\end{center}
	\caption{Initial robot configuration dialog.}
	\label{fig:platform}
\end{figure}

\subsection{Units}
\label{sec:units}
Simulations within RoboSim can be run either in {\bf US Customary} units
consisting of inches, degrees, and seconds or {\bf Metric} units with
centimeters, degrees, and seconds.  Changing units will effect the grid spacing
drawn beneath the robots and the spacing between robots.  Changing between these
two options will change the labels within the GUI to indicate the units being
used.


\subsection{Individual Robot Configuration}
Initial robot configurations can either be done through the {\bf Individual
Robot Configuration} or {\bf Preconfigured Robot Configuration} section.  The
{\bf Individual Robot Configuration} section, as shown in Figure
\ref{fig:config}, has options to allow robots to be positioned within the
RoboSim scene either with or without wheels but not attached to each other.
\begin{figure}[H]
	\begin{center}
		\includegraphics[width=6in]{images/individual}
	\end{center}
	\caption{Individual robot configuration dialog.}
	\label{fig:config}
\end{figure}

The user can specify the X and Y coordinates  as well as the orientation angle
of a virtual robot.  Images for the Linkbot and Mobot showing the meaning of
each of the options are displayed above the configuration box.  They are
screenshots of the virtual robots positioned at one foot in both the X and Y
coordinates with the orientation angle of 30 degrees from the X-axis.

Initially, the individual robot list contains one Linkbot-I at (0, 0) with 1.75
inch wheels.  More robots can be added by the 'Add Robot' button below the
configuration images.  Clicking this button will add a robot into RoboSim, each
offset from the previous one in the x-direction by 6 inches or 15 centimeters
depending upon the units selected.  The order within the robot list will be the
order in which the robots will be read into the simulation program.

\subsubsection{Robot Type}
There are three options for robot type available.  Linkbot-I, Linkbot-L,
and Mobot.  The options are presented in a drop down menu.
\begin{figure}[H]
	\begin{center}
		\includegraphics[width=6in]{images/type}
	\end{center}
	\caption{Picking a robot type.}
	\label{fig:type}
\end{figure}

\subsubsection{Robot Position}
Both X and Y positions can be chosen independently for each robot.

\subsubsection{Robot Angle}
The rotation angle from the x-axis can be used for changing the orientation of
the two robots respective to each other. 

\subsubsection{Wheels}
Since so many times the robots are run with wheels and a caster connected, a
drop down menu is provided to select different wheel sizes.  The options listed
are the radii of the wheels provided with Linkbots when purchased from Barobo.
Each wheel is drawn with a series of dots along the one radius to easily show
the rotation of the wheel.  The correlation between wheel radius and number of
dots is given in Table \ref{tab:wheels}.
\begin{table}[H]
	\begin{center}
	\begin{tabular}{c | l }
		\hline \hline
		\textbf{Number of Dots} & \textbf{Wheel Radius} \\ \hline
		2 & custom radius \\
		3 & 1.625 inch / 4.13 centimeter \\
		4 & 1.75 inch / 4.45 centimeter \\
		5 & 2.00 inch / 5.08 centimeter \\
		\hline \hline
	\end{tabular}
	\caption{Wheel sizes and number of dots.}
	\label{tab:wheels}
	\end{center}
\end{table}

Custom wheel sizes are available by using the 'Custom' option from the drop down
menu.  This option creates an input box to the right to let the user enter a
wheel radius.

\subsubsection{Remove}
A robot can be removed from the RoboSim by clicking the 'Remove' button. 

\subsection{Preconfigured Robot Configurations}
In addition to positioning robots independently within the RoboSim, some {\bf
Preconfigured Robot Configurations}, as shown in Figure \ref{fig:preconfig},
which represent commonly used Linkbot configurations  are available to the user.
Selecting one of these options will display a picture of the configuration built
with the hardware Linkbots and corresponding to a Ch robot program presented in
Chapter 13 in the book {\em Learning Robot Programming with Linkbot for the
Absolute Beginner}.  When one of these options is selected, the specific
configuration for this setup is passed into Ch and robots specified in the
individual robot configuration are ignored.  To switch back to the individual
configuration, just unselect the selected preconfigured robot configuration.
\begin{figure}[H]
	\begin{center}
		\includegraphics[width=4in]{images/preconfig}
	\end{center}
	\caption{Preconfigured Linkbots.}
	\label{fig:preconfig}
\end{figure}

\subsection{Grid Configuration}
To be able to see how far robots have moved, a grid is enabled under the robots.
There are three options to alter the layout of the grid lines under the {\bf
Grid Configuration}.  Total distance is the entire distance between -x and x for
which grid lines will be displayed.  Hashmarks are the red lines drawn within
the configuration images.  By default, the distance between two hashmarks is 12
inches in US Customary units and 50 centimeters in Metric units.  Tics are the
most frequent lines drawn in a light gray.  By default, the distance between two
tics is 1 inch in US Customary units and 5 centimeters in Metric units.

Switching between US Customary and Metric units will change these default values
to logical starting points for the metric system.  The 'Reset to Defaults' button
will allow the default values for both US Customary and Metric to be reinstated
after they have been changed.  Depending upon which units are currently selected
from Section \ref{sec:units}, either the US Customary defaults, shown in Figure
\ref{fig:grid_us}, or the Metric defaults, as shown in Figure
\ref{fig:grid_metric}, will be set.
\begin{figure}[H]
	\begin{center}
		\includegraphics[width=4in]{images/grid_us}
	\end{center}
	\caption{Default US Customary Grid Spacing.}
	\label{fig:grid_us}
\end{figure}
\begin{figure}[H]
	\begin{center}
		\includegraphics[width=4in]{images/grid_metric}
	\end{center}
	\caption{Default Metric Grid Spacing.}
	\label{fig:grid_metric}
\end{figure}

\subsection{Tracking}
{\bf Tracking} where robots have been can be enabled by selecting the check box
'Enable Robot Position Tracking', as shown in Figure \ref{fig:gui}.  When the
tracking is enabled, lines following the robot trajectories will be drawn for
each robot.  Mobot tracking lines will be in a green color and Linkbot tracking
lines will be in the color matching the Linkbot LED.

\section{Running a Ch Program with RoboSim}
Once the simulation environment has been configured with the RoboSim GUI in
Section \ref{sec:gui}, the user can run Ch programs in ChIDE to control the
virtual robots.  The RoboSim GUI should remain open while simulating robots.
Once it is closed, the system will revert to hardware mode.  The RoboSim scene
with virtual robots for each simulation are created upon running a Ch program.
For example, when the Ch program {\tt moveforward3.ch} below
\begin{verbatim}
/* File: moveforward3.ch
   Move forward for Linkbot-I as a two-wheel vehicle */
#include <linkbot.h>
CLinkbotI robot;

/* connect to the paired robot and move to the zero position */
robot.connect();
robot.resetToZero();

/* move forward by rolling two wheels for 360 degrees */
robot.moveForward(360);
\end{verbatim}
\noindent
is executed in ChIDE, a RoboSim scene shown in Figure \ref{fig:robosim_scene}
will be displayed.
\begin{verbatim}
    Paused: Press any key to start
\end{verbatim}
is displayed in the RoboSim scene to reminder the user that the virtual robot
will not move until the user presses any key on the keyboard. This gives the
user an opportunity to examine the RoboSim scene before the motion begins.
\begin{figure}[H]
	\begin{center}
		\includegraphics[width=3in]{images/robosim_scene}
	\end{center}
	\caption{A RoboSim scene with a virtual robot at its starting position.}
	\label{fig:robosim_scene}
\end{figure}

While a robot is moving in the RoboSim scene, the user can press any key to
pause the motion of the robot.  When the motion is paused, the message
\begin{verbatim}
    Paused: Press any key to restart
\end{verbatim}
will be displayed in the RoboSim scene. The user can press any key to restart
the motion.

When the user presses the 't' key, the robot trajectory is tracked in a green
line in the RoboSim scene as shown in Figure \ref{fig:robosim_tracked}.
\begin{figure}[H]
	\begin{center}
		\includegraphics[width=3in]{images/robosim_tracked}
	\end{center}
	\caption{A RoboSim scene with a virtual robot and its trajectory tracked.}
	\label{fig:robosim_tracked}
\end{figure}

When the program is finished, the message
\begin{verbatim}
    Paused: Press any key to end
\end{verbatim}
will be displayed in the RoboSim scene.  Pressing any key, the RoboSim scene
will disappear.

\section{Interacting with a RoboSim Scene}
The user can interact with a RoboSim scene through the keyboard and mouse.

The ground plane is for reference only.  It is designed to disappear when
viewing the robots from below to be able to inspect the movement from all
angles.

\subsection{Keyboard Input}
The RoboSim scene responds to keyboard input as outlined in Table
\ref{tab:keys}.  As described in the previous sections, 
the 't' key will toggle the tracking of robot trajectories.

\begin{table}[H]
	\begin{center}
	\begin{tabular}{c | l }
		\hline \hline
		\textbf{key} & \textbf{action} \\ \hline
		1 & return to home camera position \\
		2 & set camera to overhead view \\
		n & toggle grid line numbering \\
		r & toggle robot visibility and enable tracking \\
		t & toggle robot tracking \\
		any other key & Pause and unpause simulation \\
		\hline \hline
	\end{tabular}
	\caption{Keyboard input for RoboSim}
	\label{tab:keys}
	\end{center}
\end{table}

There are two views available to the user.  The default view, which can be
toggled with the '1' key, is from behind the robots looking into the first
quadrant.  This view can be seen in any of the RoboSim scene screenshots within
this document, except for Figure \ref{fig:robosim_overhead} which shows the
overhead view.  The '2' key moves the camera directly above the origin looking
down on the scene creating a 2D viewpoint of the robots.
\begin{figure}[H]
	\begin{center}
		\includegraphics[width=3in]{images/robosim_overhead}
	\end{center}
	\caption{A RoboSim scene with the overhead viewing angle.}
	\label{fig:robosim_overhead}
\end{figure}

The 'n' key allows the user to toggle the display of the grid numbering.  X and
Y numbering is by default enabled and given for every hashmark on the grid.

The 'r' key will toggle the display of virtual robots or robot trajectories.
This feature is useful when the user would like to view a trajectory traced by a
robot without the virtual root blocking the trajectory.  Figure
\ref{fig:robosim_norobot} shows a RoboSim scene with a tracked robot trajectory
only.
\begin{figure}[H]
	\begin{center}
		\includegraphics[width=3in]{images/robosim_norobot}
	\end{center}
	\caption{A RoboSim scene with a tracked robot trajectory only.}
	\label{fig:robosim_norobot}
\end{figure}

As described in the previous section, the motion of robots in the RoboSim scene
can be paused and restarted by pressing any other key on the keyboard.

\subsection{Mouse Input}
Clicking on a robot in a RoboSim scene will enable a pop up which displays the
robot number and the current position of the robot, as shown in Figure
\ref{fig:robosim_pos} with the position (0, 10.9817) inches for the X and Y
coordinates for the Robot 1.  Clicking again the displayed position for the
robot will disappear.
\begin{figure}[H]
	\begin{center}
		\includegraphics[width=3in]{images/robosim_pos}
	\end{center}
	\caption{A RoboSim scene with a virtual robot and its position displayed.}
	\label{fig:robosim_pos}
\end{figure}

The user can execute a Ch robot program in debug mode in ChIDE, line by line,
with the command {\tt Next}. At the end of each motion statement, the user can
click the robot in the RoboSim scene to obtain the X and Y coordinates of the
robot.  The ability to obtain the X and Y coordinates of a robot during its
motion along a trajectory can be very useful for learning many math concepts.

The mouse can be used to move the camera around the scene.  Holding the left
mouse button and dragging the mouse pans the camera as outlined in Table
\ref{tab:buttons}.  Holding the right mouse button and dragging the mouse
enables scaling of the view by zooming in and out.  Holding both left and right
mouse buttons and dragging changes the location of the camera within the scene.

The ground plane is for reference only.  The ground plane will disappear when
viewing the robots from below so that the user can inspect the movement from all
angles.

\begin{table}[H]
	\begin{center}
	\begin{tabular}{c | l }
		\hline \hline
		\textbf{button} & \textbf{action} \\ \hline
		Hold left mouse button and drag& rotate camera \\
		Hold right mouse button and drag& zoom in and out \\
		Hold both left and right buttons,  and drag & pan around scene \\
		Click on a robot & display the robot position\\
		\hline \hline
	\end{tabular}
	\caption{Mouse input for the RoboSim scene.}
	\label{tab:buttons}
	\end{center}
\end{table}

\newpage
\appendix
\section{Manual Configuration File Generation}
\subsection{Configuration Section}
General parameters about the simulation can be added within the config section.
Each one is its own line placed between the starting \verb%<config>% and ending
\verb%</config>%.
\begin{verbatim}
<config>
</config>
\end{verbatim}

\subsubsection{Version}
\begin{verbatim}
<version val="1"/>
\end{verbatim}
The version of the XML configuration file.  Updated internally when
new non-backward-compatible changes are made.

\subsubsection{Type}
\begin{verbatim}
<type val="0"/>
\end{verbatim}
There are two options: either preconfigured robots or individual robots.  Used
to load the right options when launching the GUI.  Can be 0 to show that the
robots within the \verb%<sim></sim>% are individual.  Above that represent the
preconfigured robots within the GUI.

\subsubsection{Grid}
\begin{verbatim}
<grid units="1" dist="48" major="12" tics="1"/>
\end{verbatim}
The grid boxes from the GUI put their information here.  \verb%units%: 1 for US
Customary and 0 for Metric.  \verb%dist% is the total distance; \verb%major% are
the red hashmarks; \verb%tics% are the gray tick marks.

\subsubsection{Tracking}
\begin{verbatim}
<tracking val="1"/>
\end{verbatim}
Setting to track robot location with lines on the ground.  1 for on; 0 for off.

\subsubsection{Buddy}
\begin{verbatim}
<buddy val="1"/>
\end{verbatim}
Setting to apply constraints based upon buddy motion.  1 for on; 0 for off.

\subsubsection{Mu}
\begin{verbatim}
<mu ground="0.9" body="0.3"/>
\end{verbatim}
The coefficient of friction can be altered between the robots and the ground and
between robots themselves.  The default values are given here.

\subsubsection{Coefficient of Restitution}
\begin{verbatim}
<cor ground="0.3" body="0.3"/>
\end{verbatim}
The coefficient of restitution can be altered between the robots and the ground
and between robots themselves.  The default values are given here.  The COR is a
measure of how bouncy a surface is.  Small values correspond to hard surfaces
while larger values are for softer surfaces.

\subsection{Simulation Section}
The simulation section holds the robots and accessories for the current
simulation.  Everything to be added is put between the \verb%<sim>% tags.
\begin{verbatim}
<sim>
</sim>
\end{verbatim}

\subsubsection{Robot Attributes}
Each robot element is required to have one attribute titled \textbf{id} which is
an unique identifier for the simulation to reference.  A second optional
attribute is \textbf{orientation} which orients the face of a second robot when
it is being attached to a first robot.  A third optional argument is
\textbf{ground} which specifies which body part of the robot is attached to the
ground.  A fixed, permanent joint is created between this body part and the
ground.

\begin{table}[H]
	\begin{center}
	\begin{tabular}{c | l}
		\hline 
		\verb|<linkboti id="0"/>| & one linkbot I with id = 0 \\
		\verb|<linkboti id="0" orientation="3"/>| & Linkbot I is 'upside-down' \\
		\hline
	\end{tabular}
	\caption{Examples}
	\label{tab:ex}
	\end{center}
\end{table}

\begin{table}[H]
	\begin{center}
	\begin{tabular}{c | c | l}
		\hline \hline
		\textbf{attribute} & \textbf{values} & \textbf{description} \\ \hline
		id & unique integer & a unique integer to identify each robot \\
		orientation & 1 & robot face number is at 12 o'clock \\
		 & 2 & robot face number is at 3 o'clock \\
		 & 3 & robot face number is at 6 o'clock \\
		 & 4 & robot face number is at 9 o'clock \\
		ground & 0 & body is attached to ground \\
		 & 1 & face 1 is attached to ground \\
		 & 2 & face 2 is attached to ground \\
		 & 3 & face 3 is attached to ground \\
		\hline \hline
	\end{tabular}
	\caption{Robot Attributes}
	\label{tab:attributes}
	\end{center}
\end{table}

\end{document}
